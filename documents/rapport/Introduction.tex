\chapter{Introduction}

\paragraph{}
Le projet relève d'une idée partagée entre quatre étudiants : créer un jeu vidéo pour téléphone mobile, de sa conception jusqu'à sa publication. Le thème du jeu de rythme a été choisi et il est développé avec le moteur de jeu \textit{Unity}. Nous nommerons ce jeu \textbf{Heliko} (escargot en Espéranto, l'escargot étant la mascotte du jeu).

\paragraph{}
La complexité du projet réside dans sa liberté. De nombreuses décisions doivent être faites pour le mener à bien et accomplir l'objectif principal : publier un produit final qui soit beau et pleinement fonctionnel. Ayant proposé notre propre sujet de TER, nous devons établir le cahier des charges initial, ainsi que concevoir en équipe le principe du jeu. Le temps est un facteur principal et difficile à gérer, puisque créer un jeu de toutes pièces en demande énormément. De plus, cela rassemble beaucoup de domaines (développement, graphisme, son, monétisation, publication...), qui nécessitent une formation lorsqu'ils ne sont pas maîtrisés.

\paragraph{}
Après avoir analysé le sujet, en parlant de l'existant et en donnant le cahier des charges, nous présenterons le rapport d'activité rendant compte des méthodes de travail que nous avons utilisées pour mener à bien ce projet. Le rapport technique décrit les choix de conception ainsi que les principaux éléments de ce qui constitue notre moteur de mini-jeux. On pourra également y trouver les outils que nous avons pu développer et des exemples concrets de l'utilisation de notre moteur. Nous conclurons en donnant les résultats, et en faisant un bilan du projet.
