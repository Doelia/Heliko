\chapter{Introduction}

\paragraph{}
Le projet relève d'une idée partagée en quatre étudiants : créer un jeu vidéo pour téléphone mobile, de sa conception jusqu'à sa publication. Le thème du un jeu de rythme a été choisit et il est développé avec le moteur de jeu \textit{Unity}. Nous nommerons ce jeu \textbf{Heliko} (Escargot en Espéranto, l'escargot étant la mascotte du jeu).

\paragraph{}
La complexité du projet réside dans sa liberté. De nombreuses décisions doivent être faites pour le mener à bien et accomplir l'objectif : Publier un jeu fonctionnel. Le cahier des charges initial n'était pas apporté et le principe du jeu devait être conçu par l'équipe. Autant de difficulté à gérer le temps car créer un jeu de toutes pièces demande énormement de temps et mélange beaucoup de domaines (Dévelopement, graphismes, sons, monétisation, publication...).

\paragraph{}
Ce rapport explique toutes les parties de la création du jeu : Conception, choix de développements, méthode de gestion de projet, outils utilisés (et outils développés), rapport technique, présentation du résultat et bilan.
