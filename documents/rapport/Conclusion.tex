\chapter{Conclusion}

\paragraph{}
La création d’un jeu vidéo de bout en bout demande énormément de temps et une grande motivation. Nous étions conscient en nous lançant dans cette tâche, que nous avons pris beaucoup de plaisir à accomplir. Seulement l’ambition ne suffit pas pour aboutir et peu conduire facilement à un échec si est trop grande.

\paragraph{}
La réalisation de ce premier jeu nous à permit de découvrir toutes les facettes et problèmes qui composent un tel projet. Beaucoup de temps a été passé pour comprendre et apprendre à utiliser les différents outils nécessaires, comme la mise en place d’une boutique ou la publication sur des marchés, que nous pensions annexes en début de projet.

\paragraph{}
Nous retenons que la création (conception et la réalisation) de contenu pour un jeu prend autant voir plus de temps que le développement pure du jeu, surtout lorsque l'on utilise un moteur de jeu comme Unity. Il est très important de prendre cela en compte lors de l’analyse, par rapport aux capacités de chaque membre d'une équipe.

\paragraph{}
Après avoir passé des tas d'heures sur le moteur Unity, nous ne regrettons pas de l'avoir choisi. Il est de plus en plus utilisé sur le marché mobile, et nous avons maintenant, grace à cette riche expérience, tous de très bonnes bases avec cet outil .

\paragraph{}
Le jeu est maintenant en production et jouable sur la plupart des téléphones du marché. Nous considérons que nous avons atteint l'objectif. Nous prendrons maintenant plaisir à analyser les statistiques et les retours des utilisateurs.

