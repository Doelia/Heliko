\chapter{Conclusion}

\paragraph{}
La création d’un jeu vidéo de bout en bout demande énormément de temps et une grande motivation. Nous en étions conscients en nous lançant dans cette tâche que nous avons pris beaucoup de plaisir à accomplir. Seulement l’ambition ne suffit pas pour aboutir et peut conduire facilement à un échec si elle est trop grande.

\paragraph{}
La réalisation de ce premier jeu nous a permis de découvrir toutes les facettes qui composent un tel projet, ainsi que leur lot de problèmes. Beaucoup de temps a été utilisé pour apprendre à utiliser les différents outils nécessaires, comme la mise en place d’une boutique ou la publication sur des marchés, que nous pensions annexes en début de projet.

\paragraph{}
Nous retenons de ce projet que la création (conception et réalisation) du contenu pour un jeu prend autant, voire plus de temps, que le développement pur du jeu, surtout lorsque l'on utilise un moteur de jeu comme Unity. Il est très important de prendre cela en compte lors de l’analyse, par rapport aux capacités de chaque membre d'une équipe.

\paragraph{}
Après avoir passé un grand nombre d'heures sur le moteur Unity, nous ne regrettons pas de l'avoir choisi. Il est de plus en plus utilisé sur le marché du mobile, et nous avons maintenant, grâce à cette riche expérience, de très bonnes bases avec cet outil .

\paragraph{}
Le jeu est maintenant en production, et jouable sur la plupart des téléphones du marché. Nous considérons que nous avons atteint l'objectif, et nous prendrons maintenant plaisir à analyser les statistiques et les retours des utilisateurs.
