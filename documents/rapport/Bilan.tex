\chapter{Bilan du projet}

\section{Autocritique}
Nous avons pu valider quatre fonctionnalités parmi celles qui étaient prévues au départ dans le cahier des charges.
\begin{itemize}
\item Créer le moteur de jeu de rythme avec Unity, et un jeu l'utilisant
\item Rendre l'application jouable sur tous les supports
\item Créer d'autres mini-jeux utilisant le moteur
\item Créer un moteur de tutoriels, et créer les tutoriels pour les jeux
\end{itemize}

\paragraph{}
Les trois premières fonctionnalités sont les plus importantes et donc nécessaires à la réussite du projet. Elles ont étés implémentées avec succès.

\paragraph{}
Cependant, plusieurs idées initiales ont du être abandonnées en cours de projet. Le principe de la "planète d'accueil" que le joueur devait personnaliser grâce aux objets qu'il aurait gagné dans les mini-jeux, ainsi que le niveau qui se joue à l'infini, n'ont pas étés développés. En effet, après avoir réalisé l'ampleur de la tâche que représentait la création d'un seul mini-jeu, de sa conception, jusqu'à son intégration finale avec le moteur, nous avons décidé de nous impliquer pleinement dans ce processus afin d'obtenir des mini-jeux ergonomiques et cohérents pour un utilisateur. C'est ainsi que plusieurs prototypes de mini-jeux n'ont pas dépassé le stade de la conception car jugés peu compréhensibles pour le type de gameplay facile et amusant que nous désirions.

\begin{figure}[H]\centering
   \begin{minipage}{0.49\textwidth}\centering
     \includegraphics[scale=0.1]{./img/concept_moulin.png}
     \caption{Concept du jeu du moulin}
     \label{jeu_concept1}
   \end{minipage}
   \begin {minipage}{0.49\textwidth}\centering
     \includegraphics[scale=0.1]{./img/concept_chevalier.png}
     \caption{Concept d'un jeu de chevaliers}
     \label{jeu_concept2}
   \end{minipage}
\end{figure}

\paragraph{}
Le principal objectif que nous nous étions fixés au sein du groupe, était d'avoir une application fonctionnelle, propre, bien codée et en ligne. Cet objectif est accompli, l'application est fonctionnelle et présente sur les marchés d'applications pour les systèmes Android et Apple.

\section{Enseignements tirés}
Les difficultés que nous avons rencontrées tout au long du développement de l'application nous ont permis de tirer des leçons et de gagner de l'expérience.

Dans un premier temps, la maitrise de l'environnement d'Unity nous a demandé beaucoup de temps, et pour cela nous avons créé de nombreuses scènes de test au démarrage du projet afin de pouvoir se familiariser avec l'outil, avant de commencer à travailler sur le véritable projet. Malgré ces précautions, nous avons rencontré quelques difficultés sur le développement principal. Ainsi, en travaillant en collaboration, via GitHub, nous avons constaté qu'il est impossible de travailler à deux en même temps sur la même scène, Unity ne sachant pas gérer le fusion de deux scène. Pour résoudre ce problème nous avons dû travailler sur des copies de scènes en leur donnant des noms de version. D'autres difficultés liées à Unity sont apparues, comme la non-compatibilité entre les versions ou la nécessité d'utiliser Windows. De plus, Unity est un logiciel demandant beaucoup de ressources, ce qui a posé des petits problèmes d'efficacité lorsque nous nous retrouvions en groupe pour travailler, avec des ordinateurs portables au capacités de calculs faibles pour travailler efficacement.

Du point de vue du travail en équipe, nous avons du apprendre à mieux connaitre les capacités de chacun afin d'être efficaces sur l'organisation. L'avantage d'avoir passé du temps à analyser le projet avant de commencer à le développer, est que nous avons pu choisir nos activités sur lesquelles nous étions chacun le plus performant.

\paragraph{}
Ce projet a été un challenge pour nous quatre, qui nous a conduits à un véritable enrichissement.

\section{Perspectives}



